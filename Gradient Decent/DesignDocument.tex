\documentclass[]{article}
\usepackage{amsmath}


%opening
\title{GPU Gradient Decent Shape Optimizer Design}
\author{}

\begin{document}
	
	\maketitle
	
	
	\section{Requirements}
	Write a GPU accelerated shape optimizer to practice GPU optimizations and investigate speed up for Morpho first version should be able to:
	\begin{itemize}
		\item Optimize surface Area with constant Volume
		\item Read in morpho meshes (for ease of use)
		\item Run on the GPU
		\item Input is Morpho mesh print area and volume. output is optimized morpho mesh print area and volume
	\end{itemize}
	
	\section{Conceptual Method}
	A mesh is a list of vertices $\vec{r_i}$ with rules for how the connect. One dimensional connections are lines, Two dimensional connections are facets. A connection is a list of vertex indices that are part of the higher dimensional element. Let $i$ label verties, $k$ label facets and $l$ label the vertex in a facet.
	Consider a facet $f_k$ with three vertices labeled $\vec{r}_{f_{k_l}}$ I.E $\vec{r}_{f_{k_1}},\vec{r}_{f_{k_2}}\vec{r}_{f_{k_3}}$. The area of this facet is:
	$$A_k = (\vec{r}_{f_{k_2}} - \vec{r}_{f_{k_1}}) \times  (\vec{r}_{f_{k_3}} - \vec{r}_{f_{k_2}})$$
	To minimize area we need to find the gradient of the area with respect to the vertices of the facet.
	\begin{align}
	\vec{S}_0 &=  (\vec{r}_{f_{k_2}} - \vec{r}_{f_{k_1}})\\
	\vec{S}_1 &= (\vec{r}_{f_{k_3}} - \vec{r}_{f_{k_2}})\\
	\vec{S}_{01} &= \vec{S}_0\times \vec{S_1}\\
	\vec{S}_{010} &= \vec{S}_{01}\times \vec{S}_0\\
	\vec{S}_{011} &= \vec{S}_{01}\times \vec{S}_1\\
	\nabla_{\vec{r}_{f_{k_1}}}A_{f_k} &=  \frac{\vec{S}_{011}}{2 |\vec{S}_{01}|}\\
	\nabla_{\vec{r}_{f_{k_2}}}A_{f_k} &=  -\frac{\vec{S}_{011}+\vec{S}_{010}}{2 |\vec{S}_{01}|}\\
	\nabla_{\vec{r}_{f_{k_3}}}A_{f_k} &=  \frac{\vec{S}_{010}}{2 |\vec{S}_{01}|}
	\end{align}
	Now that we have the area per vertex per facet we need to collate them to find the gradient. To do this on the GPU avoiding write collisions we use a map from $F(i):i\rightarrow{(k,l)_{m_i}}$ so then where $m_i$ counts up to the number of facets vertex $i$ is included in.
	$$\nabla_{\vec{r}_i} A = \sum_{(k,l)_{m_i}} \nabla_{\vec{r}_{f_{k_l}}}A_{f_k}$$
	
\end{document}

